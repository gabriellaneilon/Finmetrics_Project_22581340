\documentclass[12pt,preprint, authoryear]{elsarticle}

\usepackage{lmodern}
%%%% My spacing
\usepackage{setspace}
\setstretch{1.2}
\DeclareMathSizes{12}{14}{10}{10}

% Wrap around which gives all figures included the [H] command, or places it "here". This can be tedious to code in Rmarkdown.
\usepackage{float}
\let\origfigure\figure
\let\endorigfigure\endfigure
\renewenvironment{figure}[1][2] {
    \expandafter\origfigure\expandafter[H]
} {
    \endorigfigure
}

\let\origtable\table
\let\endorigtable\endtable
\renewenvironment{table}[1][2] {
    \expandafter\origtable\expandafter[H]
} {
    \endorigtable
}


\usepackage{ifxetex,ifluatex}
\usepackage{fixltx2e} % provides \textsubscript
\ifnum 0\ifxetex 1\fi\ifluatex 1\fi=0 % if pdftex
  \usepackage[T1]{fontenc}
  \usepackage[utf8]{inputenc}
\else % if luatex or xelatex
  \ifxetex
    \usepackage{mathspec}
    \usepackage{xltxtra,xunicode}
  \else
    \usepackage{fontspec}
  \fi
  \defaultfontfeatures{Mapping=tex-text,Scale=MatchLowercase}
  \newcommand{\euro}{€}
\fi

\usepackage{amssymb, amsmath, amsthm, amsfonts}

\def\bibsection{\section*{References}} %%% Make "References" appear before bibliography


\usepackage[round]{natbib}

\usepackage{longtable}
\usepackage[margin=2.3cm,bottom=2cm,top=2.5cm, includefoot]{geometry}
\usepackage{fancyhdr}
\usepackage[bottom, hang, flushmargin]{footmisc}
\usepackage{graphicx}
\numberwithin{equation}{section}
\numberwithin{figure}{section}
\numberwithin{table}{section}
\setlength{\parindent}{0cm}
\setlength{\parskip}{1.3ex plus 0.5ex minus 0.3ex}
\usepackage{textcomp}
\renewcommand{\headrulewidth}{0.2pt}
\renewcommand{\footrulewidth}{0.3pt}

\usepackage{array}
\newcolumntype{x}[1]{>{\centering\arraybackslash\hspace{0pt}}p{#1}}

%%%%  Remove the "preprint submitted to" part. Don't worry about this either, it just looks better without it:
\makeatletter
\def\ps@pprintTitle{%
  \let\@oddhead\@empty
  \let\@evenhead\@empty
  \let\@oddfoot\@empty
  \let\@evenfoot\@oddfoot
}
\makeatother

 \def\tightlist{} % This allows for subbullets!

\usepackage{hyperref}
\hypersetup{breaklinks=true,
            bookmarks=true,
            colorlinks=true,
            citecolor=blue,
            urlcolor=blue,
            linkcolor=blue,
            pdfborder={0 0 0}}


% The following packages allow huxtable to work:
\usepackage{siunitx}
\usepackage{multirow}
\usepackage{hhline}
\usepackage{calc}
\usepackage{tabularx}
\usepackage{booktabs}
\usepackage{caption}


\newenvironment{columns}[1][]{}{}

\newenvironment{column}[1]{\begin{minipage}{#1}\ignorespaces}{%
\end{minipage}
\ifhmode\unskip\fi
\aftergroup\useignorespacesandallpars}

\def\useignorespacesandallpars#1\ignorespaces\fi{%
#1\fi\ignorespacesandallpars}

\makeatletter
\def\ignorespacesandallpars{%
  \@ifnextchar\par
    {\expandafter\ignorespacesandallpars\@gobble}%
    {}%
}
\makeatother

\newenvironment{CSLReferences}[2]{%
}

\urlstyle{same}  % don't use monospace font for urls
\setlength{\parindent}{0pt}
\setlength{\parskip}{6pt plus 2pt minus 1pt}
\setlength{\emergencystretch}{3em}  % prevent overfull lines
\setcounter{secnumdepth}{5}

%%% Use protect on footnotes to avoid problems with footnotes in titles
\let\rmarkdownfootnote\footnote%
\def\footnote{\protect\rmarkdownfootnote}
\IfFileExists{upquote.sty}{\usepackage{upquote}}{}

%%% Include extra packages specified by user
\usepackage{amsmath}\usepackage{booktabs}
\usepackage{caption}
\usepackage{longtable}
\usepackage{colortbl}
\usepackage{array}

%%% Hard setting column skips for reports - this ensures greater consistency and control over the length settings in the document.
%% page layout
%% paragraphs
\setlength{\baselineskip}{12pt plus 0pt minus 0pt}
\setlength{\parskip}{12pt plus 0pt minus 0pt}
\setlength{\parindent}{0pt plus 0pt minus 0pt}
%% floats
\setlength{\floatsep}{12pt plus 0 pt minus 0pt}
\setlength{\textfloatsep}{20pt plus 0pt minus 0pt}
\setlength{\intextsep}{14pt plus 0pt minus 0pt}
\setlength{\dbltextfloatsep}{20pt plus 0pt minus 0pt}
\setlength{\dblfloatsep}{14pt plus 0pt minus 0pt}
%% maths
\setlength{\abovedisplayskip}{12pt plus 0pt minus 0pt}
\setlength{\belowdisplayskip}{12pt plus 0pt minus 0pt}
%% lists
\setlength{\topsep}{10pt plus 0pt minus 0pt}
\setlength{\partopsep}{3pt plus 0pt minus 0pt}
\setlength{\itemsep}{5pt plus 0pt minus 0pt}
\setlength{\labelsep}{8mm plus 0mm minus 0mm}
\setlength{\parsep}{\the\parskip}
\setlength{\listparindent}{\the\parindent}
%% verbatim
\setlength{\fboxsep}{5pt plus 0pt minus 0pt}



\begin{document}



\begin{frontmatter}  %

\title{Optimizing Portfolio Strategies: A Comparative Study of
Mean-Variance, Minimum-Variance, and Risk Parity Portfolios}

% Set to FALSE if wanting to remove title (for submission)




\author[Add1]{Gabriella Neilon}
\ead{22581340@sun.ac.za}





\address[Add1]{Stellenbosch University}

\cortext[cor]{Corresponding author: Gabriella Neilon}

\begin{abstract}
\small{
This study conducts a comprehensive examination of portfolio
optimization strategies, focusing on Mean-Variance Optimization (MVO),
Minimum-Variance Portfolio (MVP), and Risk Parity Portfolios.
Acknowledging the foundational principles of Modern Portfolio Theory
(MPT), the analysis critiques MVO's concentration issues and MVP's
struggles with uneven weight distributions. Subsequently, the study
explores the Risk Parity Portfolio's potential in mitigating asset
concentration and optimizing risk diversification. The paper utilizes
various indices to construct portfolios, employing a range of
optimization techniques and considering global assumptions. The results
reveal nuanced insights into each portfolio's performance, risk-adjusted
returns, and weight allocations across asset classes. This comparative
study highlights the strengths and weaknesses of each strategy, shedding
light on their suitability in diverse market conditions.
}
\end{abstract}

\vspace{1cm}





\vspace{0.5cm}

\end{frontmatter}

\setcounter{footnote}{0}


\renewcommand{\contentsname}{Table of Contents}
{\tableofcontents}

%________________________
% Header and Footers
%%%%%%%%%%%%%%%%%%%%%%%%%%%%%%%%%
\pagestyle{fancy}
\chead{}
\rhead{}
\lfoot{}
\rfoot{\footnotesize Page \thepage}
\lhead{}
%\rfoot{\footnotesize Page \thepage } % "e.g. Page 2"
\cfoot{}

%\setlength\headheight{30pt}
%%%%%%%%%%%%%%%%%%%%%%%%%%%%%%%%%
%________________________

\headsep 35pt % So that header does not go over title




\newpage

\newpage

\hypertarget{introduction}{%
\section*{Introduction}\label{introduction}}
\addcontentsline{toc}{section}{Introduction}

The inception of Modern Portfolio Theory (MPT) by Markowitz in 1952
revolutionized portfolio optimization, emphasizing the dual goals of
maximizing returns while minimizing risks. However, the application of
MPT strategies such as Mean-Variance Optimization (MVO) and
Minimum-Variance Portfolio (MVP) has faced criticism due to
concentration issues and challenges in risk diversification. To address
these concerns, Risk Parity Portfolios emerged as an alternative
approach, aiming to balance risk contributions across assets. This paper
investigates and compares these strategies, aiming to provide a
comprehensive understanding of their efficacy and limitations in modern
financial landscapes.

\hypertarget{literature-review}{%
\section{\texorpdfstring{Literature Review
\label{-}}{Literature Review }}\label{literature-review}}

\hypertarget{mean-variance-optimization-mvo}{%
\subsection{Mean-Variance Optimization
(MVO)}\label{mean-variance-optimization-mvo}}

The concept of \emph{Modern Portfolio Theory} traces its origins to the
pioneering work by Markowitz in 1952. This theory emphasizes the dual
objectives of maximizing expected returns while minimizing risks,
standing as the cornerstone in portfolio optimization
(\protect\hyperlink{ref-markowits1952portfolio}{Markowits, 1952}). The
approach employs the fundamental principles of portfolio diversification
and leverages covariance relationships. Portfolio diversification,
integral to risk reduction, is crucial as portfolio investment risk
(variance) is impacted by both individual asset variances and
covariances in a bipartite manner. However, the risk mitigation aspect
of portfolio diversification should be approached judiciously, as its
primary potential lies in mitigating unsystematic risk rather than
systematic risk.

The primary aim of mean-variance optimization is to minimize portfolio
volatility while targeting an expected return. However, mean-variance
optimization often leads to highly concentrated portfolios, making them
vulnerable to abrupt changes in allocations due to minor input
variations. There exists confusion between optimizing volatility and
risk diversification (\protect\hyperlink{ref-bruder2012managing}{Bruder
\& Roncalli, 2012}).

Despite its foundational principles, mean--variance optimization (MVO)
faces scrutiny when confronted with contemporary challenges,
particularly financial crises. Critics argue that it overlooks risk
diversification and only takes the portfolio's overall risk into
account, which results in an excessive concentration of risk in a small
number of assets---as was shown during the 2008 financial crisis. This
critique, additionally, questions the reliance of Modern Portfolio
Theory (MPT) on historical data as is also it is highly sensitive to
parameter estimation errors , suggesting its potential irrelevance in
current and future markets. Consequently, MPT's predictive capacity
becomes less dependable and more susceptible to deviations from actual
market behavior
(\protect\hyperlink{ref-steinbach2001markowitz}{Steinbach, 2001}).
Another area of criticism pertains to Markowitz's definition of risk,
often equated with ``volatility'' (both upside and downside). This
overlooks the perspective that investors aren't inherently risk-averse;
rather, they exhibit a tendency toward aversion to losses, therefore in
actuality, the variance is a poor indicator of risk because it penalizes
both desired low losses and unwanted large losses. As Harold Evensky, a
renowned financial planner and the founder of Evensky \& Katz Foldes
Wealth Management, states, \emph{``Investors aren't risk-averse, they're
loss-averse.''}

\hypertarget{minimum-variance-portfolio-mvp}{%
\subsection{Minimum-Variance Portfolio
(MVP)}\label{minimum-variance-portfolio-mvp}}

The Minimum-Variance Portfolio (MVP) seeks to create a portfolio with
the lowest possible risk among a set of assets without emphasizing
explicit return predictions, contrasting with the Mean-Variance
Optimization (MVO) method.

In the MVP, the primary aim is to allocate weights to assets to minimize
overall portfolio variance. However, this approach tends to concentrate
on low-volatility assets, resulting in less diversified portfolios with
uneven weight distributions
(\protect\hyperlink{ref-lohre2014diversifying}{Lohre, Opfer \& Orszag,
2014}).

This portfolio is at the left-most end of the mean-variance efficient
frontier possessing the unique trait of having security weights
independent of expected returns on individual securities. Although all
portfolios on the efficient frontier aim to minimize risk for a given
return, the minimum-variance portfolio achieves this without considering
expected returns directly
(\protect\hyperlink{ref-clarke2013risk}{Clarke, De Silva \& Thorley,
2013}).

Despite its potential benefits, minimum-variance portfolios commonly
struggle with concentration issues
(\protect\hyperlink{ref-maillard2010properties}{Maillard, Roncalli \&
Teïletche, 2010}). Ans across different portfolio components.

\hypertarget{risk-parity-portfolios}{%
\subsection{Risk Parity Portfolios}\label{risk-parity-portfolios}}

The criticism of MVO and MVP introduces alternative portfolio
optimization strategies, with a particular focus on risk budgeting (or
diversified risk parity strategies). This strategy is widely accepted in
both academic and professional circles
(\protect\hyperlink{ref-bruder2012managing}{Bruder \& Roncalli, 2012};
\protect\hyperlink{ref-choueifaty2008toward}{Choueifaty \& Coignard,
2008}; \protect\hyperlink{ref-lohre2012diversified}{Lohre, Neugebauer \&
Zimmer, 2012}; \protect\hyperlink{ref-maillard2010properties}{Maillard
\emph{et al.}, 2010}; \protect\hyperlink{ref-meucci2005risk}{Meucci,
2005}, \protect\hyperlink{ref-meucci2009managing}{2009}).

The resolution to concentrated risk in a select few assets observed in
both Mean-Variance Optimization (MVO) and Minimum Variance Portfolio
(MVP) strategies appears to lie in the adoption of a Risk Parity
portfolio. Maillard \emph{et al.}
(\protect\hyperlink{ref-maillard2010properties}{2010}) delves into the
theoretical properties of the risk budgeting portfolio, demonstrating
its volatility positioning between the minimum variance and weight
budgeting portfolios Unlike the minimum variance portfolio, the Risk
Parity portfolio is invested in all assets. While its volatility is
greater than the minimum variance portfolio but smaller than the 1/N
strategy, the Risk Parity portfolio maintains more balanced risk
contributions, even though it ranks similarly to a MVP in terms of
weight distribution.

In contrast to minimum-variance portfolios that equalize the marginal
contributions of each asset to portfolio risk, risk parity portfolios
equalize the total risk contribution, minimum-variance portfolios,
positioning risk parity portfolios slightly inside the efficient
frontier rather than on it
(\protect\hyperlink{ref-clarke2013risk}{Clarke \emph{et al.}, 2013:
40}).

And unlike Mean-Variance Optimization (MVO), risk parity portfolios do
not explicitly prioritize the expected return or the risk of a
portfolio. Nonetheless, they do necessitate a positive expected return
(\protect\hyperlink{ref-fisher2015risk}{Fisher, Maymin \& Maymin, 2015:
42}). Furthermore, the purpose of risk parity is not solely to minimize
portfolio risk like standard MVO portfolios. Instead, by equalizing
asset risk contributions, risk parity aims for optimal risk
diversification (\protect\hyperlink{ref-costa2020robust}{Costa \& Kwon,
2020}).

Initially, a risk parity portfolio defined weights based on asset class
inverse volatility, disregarding their correlations
(\protect\hyperlink{ref-clarke2013risk}{Clarke \emph{et al.}, 2013:
39}). Subsequently, Qian
(\protect\hyperlink{ref-qian2005financial}{2005}) developed a more
comprehensive definition that considers correlations and grounds the
property in a risk budget where weights are adjusted to ensure each
asset contributes equally to portfolio risk.

They allocate weights to different asset classes based on their risk
measures, ensuring each asset contributes an equal risk amount to the
portfolio and reducing estimation noise
(\protect\hyperlink{ref-maillard2010properties}{Maillard \emph{et al.},
2010}). This approach emphasizes risk allocation, typically defined as
volatility, rather than capital allocation. By adjusting asset
allocations to the same risk level, the risk parity portfolio can
achieve a higher Sharpe ratio and better withstand market downturns
compared to traditional portfolios. It ensures a well-diversified
portfolio by requiring each asset to contribute the same level of risk
(\protect\hyperlink{ref-merton1980estimating}{Merton, 1980}).

Furthermore, Ardia, Bolliger, Boudt \& Gagnon-Fleury
(\protect\hyperlink{ref-ardia2017impact}{2017}) demonstrate that risk
parity portfolios are less susceptible to covariance misspecification
compared to other risk-based investment strategies.

Nevertheless, the Risk Parity portfolio is not devoid of shortcomings.
Maillard \emph{et al.}
(\protect\hyperlink{ref-maillard2010properties}{2010: 16}) points out
that even though minimum variance portfolios face constraints due to
asset concentration, risk parity portfolios lack adequate risk
oversight. However, implementing this approach optimizes risk
distribution, acting as a balanced risk filter that prevents any single
asset from dominating the portfolio, as observed in the case of the MVP.

\hypertarget{data-and-methodology}{%
\section{\texorpdfstring{Data and Methodology
\label{-}}{Data and Methodology }}\label{data-and-methodology}}

This paper aligns closely with Maillard \emph{et al.}
(\protect\hyperlink{ref-maillard2010properties}{2010}) by constructing a
G\emph{lobal diversified portfolio} and utilizes monthly data provided
by Katzke (\protect\hyperlink{ref-katzke2023portfoliocourse}{2023a}),
focusing on various indices including the Morgan Stanley Capital
International All Country World Index (``MSCI ACWI''), the Gold Spot
rate (``Gold Spot \$/Oz''), the Bloomberg Global Aggregate Bond Index
(``GlobalAgg Unhedged USD''), the Capped SWIX All Share Index
(``J433''), and the FTSE/JSE All Bond Index (``ALBI''). The analysis
aims to compare various portfolio optimization strategies. Utilizing a
standard shrinkage technique, covariance matrices are estimated for all
strategies (\protect\hyperlink{ref-chaves2011risk}{Chaves, Hsu, Li \&
Shakernia, 2011}).

The study adopts several global assumptions: (1) full investment
(\(\boldsymbol{\omega^1} 1=1\)), (2) a long-only
strategy(\(\boldsymbol{\omega} \ge 0\)). All portfolios undergo
quarterly rebalancing, with an upper limit of 25\% and a lower bound of
1\% applied to all asset classes. Furthermore, Bonds are capped at 25\%,
Equities at 60\%, and Gold at 10\% exposure.

The analysis commences with a \textnormal{na\"{i}ve} \(\frac{1}{N}\) ,
assigning equal weights to asset classes. Subsequently, Minimum Variance
Optimization (MVO) and Minimum Variance Portfolio (MVP) techniques are
implemented. Equations representing the minimum variance portfolios are
detailed, encompassing both MVP and MVO, as provided by
(\protect\hyperlink{ref-katzke2023portfolio}{Katzke, 2023b}):

\begin{align}
\frac{1}{2} \boldsymbol{\omega^T*Dmat*\omega}=\boldsymbol{dvec^T\omega} \\
s.t. \ \boldsymbol{Amat * \omega \ge bvec}\\
\end{align}

\begin{align}
 \boldsymbol{\omega_{mvp}}= arg \min \boldsymbol{\omega' \Sigma\omega} \\
  s.t. \sum_{n=1}^{N} \omega_i = 1 \ and\ \omega \ge 0
\end{align}

\begin{align}
 \boldsymbol{\omega_{mvp}}= arg \min \boldsymbol{\omega' \Sigma\omega} \\
  s.t. \sum_{n=1}^{N} \omega_i = 1 \ and\ \omega \ge 0
\end{align}

\begin{align}
 \boldsymbol{\omega_{mvo}}= arg\max\ \ \boldsymbol{\mu}^T\omega-\lambda\boldsymbol{\omega^T\Sigma\omega} \\
  s.t. \sum_{n=1}^{N} \omega_i = 1 \ and\ \omega \ge 0
\end{align}

To ensure robustness against outliers in the dataset, a covariance
matrix shrinkage approach, advocated by {[}Ledoit \& Wolf
(\protect\hyperlink{ref-ledoit2003improved}{2003}), is applied.
Shrinkage serves to minimize the impact of outliers, enhancing the
stability of covariance estimates.

\begin{align}
\hat{\Sigma}^{Shrunk}=(1-\rho)\ *  \hat{\Sigma} \ + \rho \ * T\\
with \ T= \frac{1}{N}Tr \times I
\end{align}

Moving to risk parity, the portfolio setup involves optimizing risk
contribution through weight allocations, ensuring proportional risk
distribution among assets. The process for equalizing risk contributions
is delineated, highlighting the attempt to balance risks across the
portfolio (\protect\hyperlink{ref-RPP}{Vinícius \& Palomar, 2019}).

The volatility of the portfolio
\(\sigma(\boldsymbol{w})=\sqrt{\boldsymbol{w^t\Sigma w}}\), the risk
contribution of the \(i\)th asset to the total risk is defined as:

\(RC_i=\frac{w_i((\boldsymbol{\Sigma w}))i}{\sqrt{\boldsymbol{w^t\Sigma w}}}\)
which satisfies \(\sum_{i=1}^{N} RC_i = \sigma(\boldsymbol{w})\). The
relative or marginal risk contribution (RRC) is a normalized version so
that \(\sum_{i=1}^{N} RRC_i = 1\).

Therefore the risk parity profile which attempts to equalize the risk
contributions is: \(RC_i=\frac{1}{N}\sigma (\boldsymbol{w})\)

The risk budget constraint attempts to allocate the risk according to
the risk profile determined by the weights \(\boldsymbol{b}\) (with
\(\boldsymbol{1}^T \boldsymbol{b} =1\) and \(\boldsymbol{b} \ge 0\)) so
that it can be expressed as \(RC_i=b_i\sigma (\boldsymbol{w})\)

For the \textnormal{na\"{i}ve} diagonal formulation the constraints are
\(\boldsymbol{1}^T \boldsymbol{w} =1\) and \(\boldsymbol{w} \ge 0\))
which will state that the portfolio is inversely proportional to the
assets' volatilities. Mathematically,

\begin{align}
\boldsymbol{w} =\frac{\boldsymbol{\sigma}^{-1}}{\boldsymbol{1}^T \boldsymbol{\sigma}^{-1}}
\end{align}

where \(\boldsymbol{\sigma^2}=Diag(\boldsymbol{\Sigma})\). Therefore,
lower weights are given to high volatility assets and higher weights to
low volatility assets. This is often referred to as ``equal volatility''
portfolio. Creating an ``equal volatility'' portfolio involves selecting
and assigning weights to individual assets in such a way that, when
combined, each asset's volatility contributes equally to the total
volatility of the portfolio. This approach aims to mitigate the impact
of any single asset's volatility on the overall risk of the portfolio,
potentially providing a more diversified and risk-balanced investment
strategy. \(sd(w_i r_i)=w_i \sigma_i=\frac{1}{N}\)

Adding the additional constraints we get:

\begin{align}
\min_{w}\ R(\boldsymbol{w}) + \lambda F(\boldsymbol{w})  \\
s.t.\ \boldsymbol{Cw}=\boldsymbol{c}, \boldsymbol{Dw}\le\boldsymbol{d}
\end{align}

\hypertarget{results}{%
\section{\texorpdfstring{Results \label{-}}{Results }}\label{results}}

In comparison to other portfolios, the \(\frac{1}{N}\) portfolio is
outperformed by the MVP and MVO in terms of the Sharpe Ratio. On the
other hand, the Risk Parity Portfolio, along with the MVO, demonstrates
superior annualized returns, consistent with the findings of Maillard
\emph{et al.} (\protect\hyperlink{ref-maillard2010properties}{2010}).
Notably, the Risk Parity Portfolio demonstrates extensive weight
allocations across most asset classes. However, a noteworthy observation
emerges in the allocation pattern regarding the `ALBI' asset class.
Despite its relatively lower risk contribution, this class holds a
significant weight in the portfolio, showcasing the intuitive nature of
the Risk Parity strategy. In contrast, the MVO aggressively targets
equities and gold but overlooks `ALBI.'

Nevertheless, the considerable weight allocation towards equities in the
Risk Parity Portfolio raises questions. This anomaly might be attributed
to the chosen `asset' classes, primarily composed of indexes and hence
exhibiting more stability. However, in a `real-life' portfolio
comprising individual assets, this allocation might differ. This aligns
with @Chaves \emph{et al.}
(\protect\hyperlink{ref-chaves2011risk}{2011})'s insight, indicating
Risk Parity's sensitivity to asset selection, which remains more art
than science.

Across portfolio management, Mean-Variance Optimization (MVO) and Risk
Parity strategies emerge as prominent contenders in cumulative returns
and annualized performance. However, when assessing portfolios based on
risk-adjusted performance, the Minimum-Variance Portfolio (MVP) and
Mean-Variance strategy take the lead, particularly excelling in the
Sharpe ratio. This ratio, which balances return against risk, places
these portfolios at the forefront in delivering optimal returns
considering inherent investment risks, aligning with Chaves \emph{et
al.} (\protect\hyperlink{ref-chaves2011risk}{2011})'s conclusions.

Although the Risk Parity Portfolio boasts substantial annualized
returns, it lags in the Sharpe ratio department.

Upon analyzing risk distribution across principal portfolios, the
\(\frac{1}{N}\) strategy predominantly budgets risk toward equities,
implying a high equity risk. Furthermore, it aligns with the typical
weight distribution of minimum variance, heavily concentrated in
low-risk asset classes like bonds, as found in
(\protect\hyperlink{ref-lohre2014diversifying}{Lohre \emph{et al.},
2014}).

\begin{figure}[H]

{\centering \includegraphics{Finmetrics_Project_files/figure-latex/Figure1-1} 

}

\caption{Relative risk contribution and Weights allocation of different portfolios \label{Figure1}}\label{fig:Figure1}
\end{figure}

\begin{figure}[H]

{\centering \includegraphics{Finmetrics_Project_files/figure-latex/Figure2-1} 

}

\caption{Cumulative Returns \label{Figure2}}\label{fig:Figure2}
\end{figure}

\begin{longtable}{lrr}
\caption*{
{\large Portfolio Performance}
} \\ 
\toprule
Portfolio & Returns (Ann.) & Sharpe Ratio (Ann.) \\ 
\midrule\addlinespace[2.5pt]
Equally Weighted & 0.09 & 0.49 \\ 
Minimum-Variance & 0.09 & 0.5 \\ 
Mean-Variance & 0.1 & 0.56 \\ 
Risk Parity & 0.1 & 0.46 \\ 
\bottomrule
\end{longtable}

\hypertarget{conclusion}{%
\section*{Conclusion}\label{conclusion}}
\addcontentsline{toc}{section}{Conclusion}

In this comparative analysis, I scrutinized Mean-Variance Optimization
(MVO), Minimum-Variance Portfolio (MVP), and Risk Parity Portfolios. The
findings unveiled the strengths and weaknesses inherent in each
strategy. While MVO and MVP demonstrated superior risk-adjusted
performance and concentration issues, the Risk Parity Portfolio
exhibited balanced risk contributions across assets, albeit with certain
sensitivity to asset selection. The study highlights the need for a
nuanced approach in portfolio optimization, considering both returns and
risk diversification. Such insights contribute significantly to
navigating the complexities of contemporary financial markets and aid in
constructing more robust investment strategies.

\hfill

\newpage

\newpage

\hypertarget{references}{%
\section*{References}\label{references}}
\addcontentsline{toc}{section}{References}

\hypertarget{refs}{}
\begin{CSLReferences}{1}{0}
\leavevmode\vadjust pre{\hypertarget{ref-ardia2017impact}{}}%
Ardia, D., Bolliger, G., Boudt, K. \& Gagnon-Fleury, J.-P. 2017. The
impact of covariance misspecification in risk-based portfolios.
\emph{Annals of Operations Research}. 254:1--16.

\leavevmode\vadjust pre{\hypertarget{ref-bruder2012managing}{}}%
Bruder, B. \& Roncalli, T. 2012. Managing risk exposures using the risk
budgeting approach. \emph{Available at SSRN 2009778}.

\leavevmode\vadjust pre{\hypertarget{ref-chaves2011risk}{}}%
Chaves, D., Hsu, J., Li, F. \& Shakernia, O. 2011. Risk parity portfolio
vs. Other asset allocation heuristic portfolios. \emph{Journal of
Investing}. 20(1):108.

\leavevmode\vadjust pre{\hypertarget{ref-choueifaty2008toward}{}}%
Choueifaty, Y. \& Coignard, Y. 2008. Toward maximum diversification.
\emph{The Journal of Portfolio Management}. 35(1):40--51.

\leavevmode\vadjust pre{\hypertarget{ref-clarke2013risk}{}}%
Clarke, R., De Silva, H. \& Thorley, S. 2013. Risk parity, maximum
diversification, and minimum variance: An analytic perspective.
\emph{The Journal of Portfolio Management}. 39(3):39--53.

\leavevmode\vadjust pre{\hypertarget{ref-costa2020robust}{}}%
Costa, G. \& Kwon, R. 2020. A robust framework for risk parity
portfolios. \emph{Journal of Asset Management}. 21(5):447--466.

\leavevmode\vadjust pre{\hypertarget{ref-fisher2015risk}{}}%
Fisher, G.S., Maymin, P.Z. \& Maymin, Z.G. 2015. Risk parity optimality.
\emph{The Journal of Portfolio Management}. 41(2):42--56.

\leavevmode\vadjust pre{\hypertarget{ref-katzke2023portfoliocourse}{}}%
Katzke, N.F. 2023a.

\leavevmode\vadjust pre{\hypertarget{ref-katzke2023portfolio}{}}%
Katzke, N.F. 2023b. {[}Online{]}, Available:
\url{https://www.fmx.nfkatzke.com/posts/2020-08-07-practical-3/}.

\leavevmode\vadjust pre{\hypertarget{ref-ledoit2003improved}{}}%
Ledoit, O. \& Wolf, M. 2003. Improved estimation of the covariance
matrix of stock returns with an application to portfolio selection.
\emph{Journal of empirical finance}. 10(5):603--621.

\leavevmode\vadjust pre{\hypertarget{ref-lohre2012diversified}{}}%
Lohre, H., Neugebauer, U. \& Zimmer, C. 2012. Diversified risk parity
strategies for equity portfolio selection. \emph{The Journal of
Investing}. 21(3):111--128.

\leavevmode\vadjust pre{\hypertarget{ref-lohre2014diversifying}{}}%
Lohre, H., Opfer, H. \& Orszag, G. 2014. Diversifying risk parity.
\emph{Journal of Risk}. 16(5):53--79.

\leavevmode\vadjust pre{\hypertarget{ref-maillard2010properties}{}}%
Maillard, S., Roncalli, T. \& Teïletche, J. 2010. The properties of
equally weighted risk contribution portfolios. \emph{The Journal of
Portfolio Management}. 36(4):60--70.

\leavevmode\vadjust pre{\hypertarget{ref-markowits1952portfolio}{}}%
Markowits, H.M. 1952. Portfolio selection. \emph{Journal of finance}.
7(1):71--91.

\leavevmode\vadjust pre{\hypertarget{ref-merton1980estimating}{}}%
Merton, R.C. 1980. On estimating the expected return on the market: An
exploratory investigation. \emph{Journal of financial economics}.
8(4):323--361.

\leavevmode\vadjust pre{\hypertarget{ref-meucci2005risk}{}}%
Meucci, A. 2005. \emph{Risk and asset allocation}. Vol. 1. Springer.

\leavevmode\vadjust pre{\hypertarget{ref-meucci2009managing}{}}%
Meucci, A. 2009. Managing diversification. \emph{Risk}. 74--79.

\leavevmode\vadjust pre{\hypertarget{ref-qian2005financial}{}}%
Qian, E.E. 2005. On the financial interpretation of risk contribution:
Risk budgets do add up. \emph{Available at SSRN 684221}.

\leavevmode\vadjust pre{\hypertarget{ref-steinbach2001markowitz}{}}%
Steinbach, M.C. 2001. Markowitz revisited: Mean-variance models in
financial portfolio analysis. \emph{SIAM review}. 43(1):31--85.

\leavevmode\vadjust pre{\hypertarget{ref-RPP}{}}%
Vinícius, Z. \& Palomar, D.P. 2019. {[}Online{]}, Available:
\url{https://cran.r-project.org/web/packages/riskParityPortfolio/vignettes/RiskParityPortfolio.html\#risk-parity-portfolio}.

\end{CSLReferences}

\bibliography{Tex/ref}





\end{document}
