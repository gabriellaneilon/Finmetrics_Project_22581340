\documentclass[12pt,preprint, authoryear]{elsarticle}

\usepackage{lmodern}
%%%% My spacing
\usepackage{setspace}
\setstretch{1.2}
\DeclareMathSizes{12}{14}{10}{10}

% Wrap around which gives all figures included the [H] command, or places it "here". This can be tedious to code in Rmarkdown.
\usepackage{float}
\let\origfigure\figure
\let\endorigfigure\endfigure
\renewenvironment{figure}[1][2] {
    \expandafter\origfigure\expandafter[H]
} {
    \endorigfigure
}

\let\origtable\table
\let\endorigtable\endtable
\renewenvironment{table}[1][2] {
    \expandafter\origtable\expandafter[H]
} {
    \endorigtable
}


\usepackage{ifxetex,ifluatex}
\usepackage{fixltx2e} % provides \textsubscript
\ifnum 0\ifxetex 1\fi\ifluatex 1\fi=0 % if pdftex
  \usepackage[T1]{fontenc}
  \usepackage[utf8]{inputenc}
\else % if luatex or xelatex
  \ifxetex
    \usepackage{mathspec}
    \usepackage{xltxtra,xunicode}
  \else
    \usepackage{fontspec}
  \fi
  \defaultfontfeatures{Mapping=tex-text,Scale=MatchLowercase}
  \newcommand{\euro}{€}
\fi

\usepackage{amssymb, amsmath, amsthm, amsfonts}

\def\bibsection{\section*{References}} %%% Make "References" appear before bibliography


\usepackage[round]{natbib}

\usepackage{longtable}
\usepackage[margin=2.3cm,bottom=2cm,top=2.5cm, includefoot]{geometry}
\usepackage{fancyhdr}
\usepackage[bottom, hang, flushmargin]{footmisc}
\usepackage{graphicx}
\numberwithin{equation}{section}
\numberwithin{figure}{section}
\numberwithin{table}{section}
\setlength{\parindent}{0cm}
\setlength{\parskip}{1.3ex plus 0.5ex minus 0.3ex}
\usepackage{textcomp}
\renewcommand{\headrulewidth}{0.2pt}
\renewcommand{\footrulewidth}{0.3pt}

\usepackage{array}
\newcolumntype{x}[1]{>{\centering\arraybackslash\hspace{0pt}}p{#1}}

%%%%  Remove the "preprint submitted to" part. Don't worry about this either, it just looks better without it:
\makeatletter
\def\ps@pprintTitle{%
  \let\@oddhead\@empty
  \let\@evenhead\@empty
  \let\@oddfoot\@empty
  \let\@evenfoot\@oddfoot
}
\makeatother

 \def\tightlist{} % This allows for subbullets!

\usepackage{hyperref}
\hypersetup{breaklinks=true,
            bookmarks=true,
            colorlinks=true,
            citecolor=blue,
            urlcolor=blue,
            linkcolor=blue,
            pdfborder={0 0 0}}


% The following packages allow huxtable to work:
\usepackage{siunitx}
\usepackage{multirow}
\usepackage{hhline}
\usepackage{calc}
\usepackage{tabularx}
\usepackage{booktabs}
\usepackage{caption}


\newenvironment{columns}[1][]{}{}

\newenvironment{column}[1]{\begin{minipage}{#1}\ignorespaces}{%
\end{minipage}
\ifhmode\unskip\fi
\aftergroup\useignorespacesandallpars}

\def\useignorespacesandallpars#1\ignorespaces\fi{%
#1\fi\ignorespacesandallpars}

\makeatletter
\def\ignorespacesandallpars{%
  \@ifnextchar\par
    {\expandafter\ignorespacesandallpars\@gobble}%
    {}%
}
\makeatother

\newenvironment{CSLReferences}[2]{%
}

\urlstyle{same}  % don't use monospace font for urls
\setlength{\parindent}{0pt}
\setlength{\parskip}{6pt plus 2pt minus 1pt}
\setlength{\emergencystretch}{3em}  % prevent overfull lines
\setcounter{secnumdepth}{5}

%%% Use protect on footnotes to avoid problems with footnotes in titles
\let\rmarkdownfootnote\footnote%
\def\footnote{\protect\rmarkdownfootnote}
\IfFileExists{upquote.sty}{\usepackage{upquote}}{}

%%% Include extra packages specified by user
\usepackage{amsmath}\usepackage{booktabs}
\usepackage{caption}
\usepackage{longtable}
\usepackage{colortbl}
\usepackage{array}

%%% Hard setting column skips for reports - this ensures greater consistency and control over the length settings in the document.
%% page layout
%% paragraphs
\setlength{\baselineskip}{12pt plus 0pt minus 0pt}
\setlength{\parskip}{12pt plus 0pt minus 0pt}
\setlength{\parindent}{0pt plus 0pt minus 0pt}
%% floats
\setlength{\floatsep}{12pt plus 0 pt minus 0pt}
\setlength{\textfloatsep}{20pt plus 0pt minus 0pt}
\setlength{\intextsep}{14pt plus 0pt minus 0pt}
\setlength{\dbltextfloatsep}{20pt plus 0pt minus 0pt}
\setlength{\dblfloatsep}{14pt plus 0pt minus 0pt}
%% maths
\setlength{\abovedisplayskip}{12pt plus 0pt minus 0pt}
\setlength{\belowdisplayskip}{12pt plus 0pt minus 0pt}
%% lists
\setlength{\topsep}{10pt plus 0pt minus 0pt}
\setlength{\partopsep}{3pt plus 0pt minus 0pt}
\setlength{\itemsep}{5pt plus 0pt minus 0pt}
\setlength{\labelsep}{8mm plus 0mm minus 0mm}
\setlength{\parsep}{\the\parskip}
\setlength{\listparindent}{\the\parindent}
%% verbatim
\setlength{\fboxsep}{5pt plus 0pt minus 0pt}



\begin{document}



\begin{frontmatter}  %

\title{Optimizing Portfolio Strategies: A Comparative Study of
Mean-Variance, Minimum-Variance, Equally Weighted, and Risk Parity
Portfolios}

% Set to FALSE if wanting to remove title (for submission)




\author[Add1]{Gabriella Neilon}
\ead{22581340@sun.ac.za}





\address[Add1]{Stellenbosch University}

\cortext[cor]{Corresponding author: Gabriella Neilon}

\begin{abstract}
\small{
This paper critically examines the effectiveness of Mean-Variance
Optimization (MVO), Minimum-Variance Portfolio (MVP), Equally Weighted
and Risk Parity portfolio strategies in the the long run and during
volatile periods. While MVO aims to maximize expected returns while
minimizing risks, it often results in concentrated portfolios vulnerable
to abrupt changes. MVP, focusing on minimizing overall portfolio
variance without explicit return predictions, faces issues of
concentration on low-volatility assets. The study introduces Risk Parity
portfolios as an alternative strategy, emphasizing risk budgeting to
achieve optimal risk diversification. The study concludes that all three
portfolio strategies consistently outperform the S\&P 500 index,
demonstrating their potential for delivering superior risk-adjusted
returns across various market conditions. Notably, Risk Parity
portfolios exhibit balanced risk contributions, making them resilient in
downturns, and the MVP strategy stands out with promising risk-adjusted
returns. The research provides valuable insights for investors seeking
optimal portfolio strategies in dynamic markets.
}
\end{abstract}

\vspace{1cm}





\vspace{0.5cm}

\end{frontmatter}

\setcounter{footnote}{0}


\renewcommand{\contentsname}{Table of Contents}
{\tableofcontents}

%________________________
% Header and Footers
%%%%%%%%%%%%%%%%%%%%%%%%%%%%%%%%%
\pagestyle{fancy}
\chead{}
\rhead{}
\lfoot{}
\rfoot{\footnotesize Page \thepage}
\lhead{}
%\rfoot{\footnotesize Page \thepage } % "e.g. Page 2"
\cfoot{}

%\setlength\headheight{30pt}
%%%%%%%%%%%%%%%%%%%%%%%%%%%%%%%%%
%________________________

\headsep 35pt % So that header does not go over title




\newpage

\newpage

\hypertarget{introduction}{%
\section*{Introduction}\label{introduction}}
\addcontentsline{toc}{section}{Introduction}

The inception of Modern Portfolio Theory (MPT) by Markowitz in 1952
revolutionized portfolio optimization, emphasizing the dual goals of
maximizing returns while minimizing risks. However, the application of
MPT strategies such as Mean-Variance Optimization (MVO) and
Minimum-Variance Portfolio (MVP) has faced criticism due to
concentration issues and challenges in risk diversification. To address
these concerns, Risk Parity Portfolios emerged as an alternative
approach, aiming to balance risk contributions across assets. This paper
investigates and compares these strategies, aiming to provide a
comprehensive understanding of their efficacy and limitations in modern
financial landscapes.

\hypertarget{literature-review}{%
\section{\texorpdfstring{Literature Review
\label{-}}{Literature Review }}\label{literature-review}}

\hypertarget{mean-variance-optimization-mvo}{%
\subsection{Mean-Variance Optimization
(MVO)}\label{mean-variance-optimization-mvo}}

The concept of \emph{Modern Portfolio Theory} traces its origins to the
pioneering work by Markowitz in 1952. This theory emphasizes the dual
objectives of maximizing expected returns while minimizing risks,
standing as the cornerstone in portfolio optimization
(\protect\hyperlink{ref-markowits1952portfolio}{Markowits, 1952}). The
approach employs the fundamental principles of portfolio diversification
and leverages covariance relationships. Portfolio diversification,
integral to risk reduction, is crucial as portfolio investment risk
(variance) is impacted by both individual asset variances and
covariances in a bipartite manner. However, the risk mitigation aspect
of portfolio diversification should be approached judiciously, as its
primary potential lies in mitigating unsystematic risk rather than
systematic risk.

The primary aim of mean-variance optimization is to minimize portfolio
volatility while targeting an expected return. However, mean-variance
optimization often leads to highly concentrated portfolios, making them
vulnerable to abrupt changes in allocations due to minor input
variations. There exists confusion between optimizing volatility and
risk diversification (\protect\hyperlink{ref-bruder2012managing}{Bruder
\& Roncalli, 2012}).

Despite its foundational principles, mean--variance optimization (MVO)
faces scrutiny when confronted with contemporary challenges,
particularly financial crises. Critics argue that it overlooks risk
diversification and only takes the portfolio's overall risk into
account, which results in an excessive concentration of risk in a small
number of assets---as was shown during the 2008 financial crisis. This
critique, additionally, questions the reliance of Modern Portfolio
Theory (MPT) on historical data as is also it is highly sensitive to
parameter estimation errors , suggesting its potential irrelevance in
current and future markets. Consequently, MPT's predictive capacity
becomes less dependable and more susceptible to deviations from actual
market behavior
(\protect\hyperlink{ref-steinbach2001markowitz}{Steinbach, 2001}).
Another area of criticism pertains to Markowitz's definition of risk,
often equated with ``volatility'' (both upside and downside). This
overlooks the perspective that investors aren't inherently risk-averse;
rather, they exhibit a tendency toward aversion to losses, therefore in
actuality, the variance is a poor indicator of risk because it penalizes
both desired low losses and unwanted large losses. As Harold Evensky, a
renowned financial planner and the founder of Evensky \& Katz Foldes
Wealth Management, states, \emph{``Investors aren't risk-averse, they're
loss-averse.''}

\hypertarget{minimum-variance-portfolio-mvp}{%
\subsection{Minimum-Variance Portfolio
(MVP)}\label{minimum-variance-portfolio-mvp}}

The Minimum-Variance Portfolio (MVP) seeks to create a portfolio with
the lowest possible risk among a set of assets without emphasizing
explicit return predictions, contrasting with the Mean-Variance
Optimization (MVO) method.

In the MVP, the primary aim is to allocate weights to assets to minimize
overall portfolio variance. However, this approach tends to concentrate
on low-volatility assets, resulting in less diversified portfolios with
uneven weight distributions
(\protect\hyperlink{ref-lohre2014diversifying}{Lohre, Opfer \& Orszag,
2014}).

This portfolio is at the left-most end of the mean-variance efficient
frontier possessing the unique trait of having security weights
independent of expected returns on individual securities. Although all
portfolios on the efficient frontier aim to minimize risk for a given
return, the minimum-variance portfolio achieves this without considering
expected returns directly
(\protect\hyperlink{ref-clarke2013risk}{Clarke, De Silva \& Thorley,
2013}).

Despite its potential benefits, minimum-variance portfolios commonly
struggle with concentration issues
(\protect\hyperlink{ref-maillard2010properties}{Maillard, Roncalli \&
Teïletche, 2010}). Ans across different portfolio components.

\hypertarget{risk-parity-portfolios}{%
\subsection{Risk Parity Portfolios}\label{risk-parity-portfolios}}

The criticism of MVO and MVP introduces alternative portfolio
optimization strategies, with a particular focus on risk budgeting (or
diversified risk parity strategies). This strategy is widely accepted in
both academic and professional circles
(\protect\hyperlink{ref-bruder2012managing}{Bruder \& Roncalli, 2012};
\protect\hyperlink{ref-choueifaty2008toward}{Choueifaty \& Coignard,
2008}; \protect\hyperlink{ref-lohre2012diversified}{Lohre, Neugebauer \&
Zimmer, 2012}; \protect\hyperlink{ref-maillard2010properties}{Maillard
\emph{et al.}, 2010}; \protect\hyperlink{ref-meucci2005risk}{Meucci,
2005}, \protect\hyperlink{ref-meucci2009managing}{2009}).

The resolution to concentrated risk in a select few assets observed in
both Mean-Variance Optimization (MVO) and Minimum Variance Portfolio
(MVP) strategies appears to lie in the adoption of a Risk Parity
portfolio. Maillard \emph{et al.}
(\protect\hyperlink{ref-maillard2010properties}{2010}) delves into the
theoretical properties of the risk budgeting portfolio, demonstrating
its volatility positioning between the minimum variance and weight
budgeting portfolios Unlike the minimum variance portfolio, the Risk
Parity portfolio is invested in all assets. While its volatility is
greater than the minimum variance portfolio but smaller than the 1/N
strategy, the Risk Parity portfolio maintains more balanced risk
contributions, even though it ranks similarly to a MVP in terms of
weight distribution.

In contrast to minimum-variance portfolios that equalize the marginal
contributions of each asset to portfolio risk, risk parity portfolios
equalize the total risk contribution, minimum-variance portfolios,
positioning risk parity portfolios slightly inside the efficient
frontier rather than on it
(\protect\hyperlink{ref-clarke2013risk}{Clarke \emph{et al.}, 2013:
40}).

And unlike Mean-Variance Optimization (MVO), risk parity portfolios do
not explicitly prioritize the expected return or the risk of a
portfolio. Nonetheless, they do necessitate a positive expected return
(\protect\hyperlink{ref-fisher2015risk}{Fisher, Maymin \& Maymin, 2015:
42}). Furthermore, the purpose of risk parity is not solely to minimize
portfolio risk like standard MVO portfolios. Instead, by equalizing
asset risk contributions, risk parity aims for optimal risk
diversification (\protect\hyperlink{ref-costa2020robust}{Costa \& Kwon,
2020}).

Initially, a risk parity portfolio defined weights based on asset class
inverse volatility, disregarding their correlations
(\protect\hyperlink{ref-clarke2013risk}{Clarke \emph{et al.}, 2013:
39}). Subsequently, Qian
(\protect\hyperlink{ref-qian2005financial}{2005}) developed a more
comprehensive definition that considers correlations and grounds the
property in a risk budget where weights are adjusted to ensure each
asset contributes equally to portfolio risk.

They allocate weights to different asset classes based on their risk
measures, ensuring each asset contributes an equal risk amount to the
portfolio and reducing estimation noise
(\protect\hyperlink{ref-maillard2010properties}{Maillard \emph{et al.},
2010}). This approach emphasizes risk allocation, typically defined as
volatility, rather than capital allocation. By adjusting asset
allocations to the same risk level, the risk parity portfolio can
achieve a higher Sharpe ratio and better withstand market downturns
compared to traditional portfolios. It ensures a well-diversified
portfolio by requiring each asset to contribute the same level of risk
(\protect\hyperlink{ref-merton1980estimating}{Merton, 1980}).

Furthermore, Ardia, Bolliger, Boudt \& Gagnon-Fleury
(\protect\hyperlink{ref-ardia2017impact}{2017}) demonstrate that risk
parity portfolios are less susceptible to covariance misspecification
compared to other risk-based investment strategies.

Nevertheless, the Risk Parity portfolio is not devoid of shortcomings.
Maillard \emph{et al.}
(\protect\hyperlink{ref-maillard2010properties}{2010: 16}) points out
that even though minimum variance portfolios face constraints due to
asset concentration, risk parity portfolios lack adequate risk
oversight. However, implementing this approach optimizes risk
distribution, acting as a balanced risk filter that prevents any single
asset from dominating the portfolio, as observed in the case of the MVP.

\hypertarget{portfolio-construction}{%
\section{Portfolio construction}\label{portfolio-construction}}

This paper aligns closely with Maillard \emph{et al.}
(\protect\hyperlink{ref-maillard2010properties}{2010}) by constructing a
\emph{Global diversified portfolio} and utilizes monthly data from
2002-02-28 to 2023-08-31 provided by Katzke
(\protect\hyperlink{ref-katzke2023portfoliocourse}{2023a}), focusing on
various indices including the Morgan Stanley Capital International All
Country World Index (``MSCI ACWI''), the Gold Spot rate (``Gold Spot
\$/Oz''), the Bloomberg Global Aggregate Bond Index (``GlobalAgg
Unhedged USD''), the Capped SWIX All Share Index (``J433''), and the
FTSE/JSE All Bond Index (``ALBI''). The analysis aims to compare various
portfolio optimization strategies. Utilizing a standard shrinkage
technique, covariance matrices are estimated for all strategies
(\protect\hyperlink{ref-chaves2011risk}{Chaves, Hsu, Li \& Shakernia,
2011}).

The study adopts several global assumptions: (1) full investment
(\(\boldsymbol{\omega^1} 1=1\)), (2) a long-only
strategy(\(\boldsymbol{\omega} \ge 0\)). All portfolios undergo
quarterly rebalancing, with an upper limit of 25\% and a lower bound of
1\% applied to all asset classes. Furthermore, Bonds are capped at 25\%,
Equities at 60\%, and Gold at 10\% exposure.

The analysis commences with a \textnormal{na\"{i}ve} \(\frac{1}{N}\) ,
assigning equal weights to asset classes. Subsequently, Minimum Variance
Optimization (MVO) and Minimum Variance Portfolio (MVP) techniques are
implemented. Equations representing the minimum variance portfolios are
detailed, encompassing both MVP and MVO, as provided by
(\protect\hyperlink{ref-katzke2023portfolio}{Katzke, 2023b}):

\begin{align}
\frac{1}{2} \boldsymbol{\omega^T*Dmat*\omega}=\boldsymbol{dvec^T\omega} \\
s.t. \ \boldsymbol{Amat * \omega \ge bvec}\\
\end{align}

\begin{align}
 \boldsymbol{\omega_{mvp}}= arg \min \boldsymbol{\omega' \Sigma\omega} \\
  s.t. \sum_{n=1}^{N} \omega_i = 1 \ and\ \omega \ge 0
\end{align}

\begin{align}
 \boldsymbol{\omega_{mvp}}= arg \min \boldsymbol{\omega' \Sigma\omega} \\
  s.t. \sum_{n=1}^{N} \omega_i = 1 \ and\ \omega \ge 0
\end{align}

\begin{align}
 \boldsymbol{\omega_{mvo}}= arg\max\ \ \boldsymbol{\mu}^T\omega-\lambda\boldsymbol{\omega^T\Sigma\omega} \\
  s.t. \sum_{n=1}^{N} \omega_i = 1 \ and\ \omega \ge 0
\end{align}

To ensure robustness against outliers in the dataset, a covariance
matrix shrinkage approach, advocated by {[}Ledoit \& Wolf
(\protect\hyperlink{ref-ledoit2003improved}{2003}), is applied.
Shrinkage serves to minimize the impact of outliers, enhancing the
stability of covariance estimates.

\begin{align}
\hat{\Sigma}^{Shrunk}=(1-\rho)\ *  \hat{\Sigma} \ + \rho \ * T\\
with \ T= \frac{1}{N}Tr \times I
\end{align}

Moving to risk parity, the portfolio setup involves optimizing risk
contribution through weight allocations, ensuring proportional risk
distribution among assets. The process for equalizing risk contributions
is delineated, highlighting the attempt to balance risks across the
portfolio (\protect\hyperlink{ref-RPP}{Vinícius \& Palomar, 2019}).

The volatility of the portfolio
\(\sigma(\boldsymbol{w})=\sqrt{\boldsymbol{w^t\Sigma w}}\), the risk
contribution of the \(i\)th asset to the total risk is defined as:

\(RC_i=\frac{w_i((\boldsymbol{\Sigma w}))i}{\sqrt{\boldsymbol{w^t\Sigma w}}}\)
which satisfies \(\sum_{i=1}^{N} RC_i = \sigma(\boldsymbol{w})\). The
relative or marginal risk contribution (RRC) is a normalized version so
that \(\sum_{i=1}^{N} RRC_i = 1\).

Therefore the risk parity profile which attempts to equalize the risk
contributions is: \(RC_i=\frac{1}{N}\sigma (\boldsymbol{w})\)

The risk budget constraint attempts to allocate the risk according to
the risk profile determined by the weights \(\boldsymbol{b}\) (with
\(\boldsymbol{1}^T \boldsymbol{b} =1\) and \(\boldsymbol{b} \ge 0\)) so
that it can be expressed as \(RC_i=b_i\sigma (\boldsymbol{w})\)

For the \textnormal{na\"{i}ve} diagonal formulation the constraints are
\(\boldsymbol{1}^T \boldsymbol{w} =1\) and \(\boldsymbol{w} \ge 0\))
which will state that the portfolio is inversely proportional to the
assets' volatilities. Mathematically,

\begin{align}
\boldsymbol{w} =\frac{\boldsymbol{\sigma}^{-1}}{\boldsymbol{1}^T \boldsymbol{\sigma}^{-1}}
\end{align}

where \(\boldsymbol{\sigma^2}=Diag(\boldsymbol{\Sigma})\). Therefore,
lower weights are given to high volatility assets and higher weights to
low volatility assets. This is often referred to as ``equal volatility''
portfolio. Creating an ``equal volatility'' portfolio involves selecting
and assigning weights to individual assets in such a way that, when
combined, each asset's volatility contributes equally to the total
volatility of the portfolio. This approach aims to mitigate the impact
of any single asset's volatility on the overall risk of the portfolio,
potentially providing a more diversified and risk-balanced investment
strategy. \(sd(w_i r_i)=w_i \sigma_i=\frac{1}{N}\)

Adding the additional constraints we get:

\begin{align}
\min_{w}\ R(\boldsymbol{w}) + \lambda F(\boldsymbol{w})  \\
s.t.\ \boldsymbol{Cw}=\boldsymbol{c}, \boldsymbol{Dw}\le\boldsymbol{d}
\end{align}

\hypertarget{results}{%
\section{\texorpdfstring{Results \label{-}}{Results }}\label{results}}

\hypertarget{preliminary-analysis}{%
\subsection{Preliminary analysis}\label{preliminary-analysis}}

This section examines the risk contribution, weight allocation, and
cumulative returns of each portfolio. While offering insights into
expected portfolio behavior, it is essential to recognize that this
analysis alone may not comprehensively assess portfolio performance. A
more robust and detailed analysis follows this section.

Figure \ref{Figure1} illustrates the expected weight allocation of the
Risk Parity portfolio, aligning with its relative risk contributions. In
contrast, the Minimum Variance Portfolio (MVP) portfolio exhibits an
inverse relationship, with a greater risk contribution corresponding to
a smaller weight allocation. A similar trend is observed in the MVO
portfolio, with variations. The Risk Parity Portfolio, notable for its
diverse allocation, particularly allocates a significant weight to the
`ALBI' asset class, showcasing the intuitive nature of the Risk Parity
strategy. Conversely, the MVP portfolio, while aggressively targeting
equities and gold, overlooks `J433' and aligns with the traditional
weight distribution of minimum variance, concentrating heavily on
low-risk asset classes such as bonds
(\protect\hyperlink{ref-lohre2014diversifying}{Lohre \emph{et al.},
2014}).

Figure \ref{Figure1} also provides insights into the investment risk
profiles of different portfolios, revealing key exposures for each asset
class. Global bonds face currency risk, while local equities are subject
to market risk influenced by economic conditions, geopolitical events,
and investor sentiment. Global equities are exposed to global market
movements, economic conditions, currency risk, and political and
regulatory risk. The Risk Parity portfolio demonstrates the highest
exposure to both local and global equities, indicating sensitivity to
global market dynamics. In contrast, the MVP Portfolio has a greater
emphasis on global and local bonds.

A higher weight allocation to bonds in a portfolio offers several
benefits. Bonds are generally less volatile than equities, contributing
to a more stable portfolio value, which is suitable for risk-averse
investors. Bonds also provide regular interest payments, creating a
predictable income stream. Additionally, bonds are considered safer in
terms of capital preservation, offering a hedge during equity market
downturns. The lower correlation between bonds and equities contributes
to diversification, reducing overall portfolio risk.

On the other hand, the Mean-Variance Optimization (MVO) exhibits a
higher exposure to the gold spot rate and local stocks. This suggests
that gold serves as a potential protective buffer, employed for hedging
against inflation risk and mitigating market and economic stress.

Figure \ref{Figure2} compares cumulative returns across portfolio
strategies, highlighting MVO and Risk Parity as prominent contenders.

\begin{figure}[H]

{\centering \includegraphics{Finmetrics_Project_files/figure-latex/Figure1-1} 

}

\caption{Relative risk contribution and Weights allocation of different portfolios \label{Figure1}}\label{fig:Figure1}
\end{figure}

\begin{figure}[H]

{\centering \includegraphics{Finmetrics_Project_files/figure-latex/Figure2-1} 

}

\caption{Cumulative Returns \label{Figure2}}\label{fig:Figure2}
\end{figure}

\hypertarget{further-analysis}{%
\subsection{Further Analysis}\label{further-analysis}}

A drawdown signifies the extent to which the portfolio declines from its
peak value before subsequently recovering. The portfolios were assessed
in relation to the S\&P 500. While the S\&P 500 may not be the ideal
benchmark, given that other portfolios are primarily constructed from
different market indices, its broad market representation,
market-capitalization weighting, diverse industry coverage, liquidity,
reputation, and global influence make it the most suitable benchmark for
this analysis.

In Figure \ref{Figure3}, several crucial metrics stand out when
dissecting this graph. These metrics include the depth, duration,
frequency, and recovery periods of drawdowns observed in various
portfolios relative S\&P 500 index. Notably, the S\&P 500 index exhibits
the most substantial depth in drawdowns, coupled with the lengthiest
recovery periods. This implies that the index not only experiences the
most significant loss in value following its peaks but also takes an
extended period to recover from these downturns. In contrast, the MVP
portfolio displays lower frequency, depth, and duration of drawdowns,
indicating comparatively smaller losses in portfolio value and
consequently lower risk compared to other portfolios.

The Risk Parity portfolio follows a similar trend to the MVP, albeit
with greater depth in drawdowns, though not as pronounced as observed in
the S\&P 500. This suggests that while the Risk Parity portfolio
experiences more significant declines in value during drawdowns, it
still maintains a risk profile that is less pronounced than that of the
S\&P 500. The \(\frac{1}{N}\) exhibits the last amoutn of depth in
drawdowns, followed by the MVo portfolio. The \(\frac{1}{N}\) portfolio
demonstrates the least depth in drawdowns, with the MVO portfolio
following closely.

\begin{figure}[H]

{\centering \includegraphics{Finmetrics_Project_files/figure-latex/Figure3-1} 

}

\caption{Drawdowns \label{Figure3}}\label{fig:Figure3}
\end{figure}

Furthermore, I conducted an analysis of the annualized risk-return
performance of each portfolio relative to the S\&P 500 over five
distinct periods. The initial period encompasses the entire dataset,
while the second period spans from October 2008, marked by a significant
drop in the S\&P 500 after reaching its peak. The third period, December
2018, witnessed another downturn attributed to factors such as Donald
Trump's trade war with China, global economic slowdown, and concerns
about the Federal Reserve's rapid interest rate hikes. Additionally, the
year 2020, marked by the onset of the COVID-19 pandemic, and January
2022, following a peak in the S\&P 500, represent subsequent periods
characterized by geopolitical tensions, the Ukraine war, extensive
COVID-related lockdowns in China, interest rate hikes by the US Federal
Reserve, and overall economic uncertainty in the US post-pandemic. These
chosen periods were deliberately selected since it encapsulated volatile
circumstances.

\begin{figure}[H]

{\centering \includegraphics{Finmetrics_Project_files/figure-latex/Figure4-1} 

}

\caption{Annualized Risk-Return performance of portfolios relative to SP 500 over the entire sample period. \label{Figure4}}\label{fig:Figure4}
\end{figure}

Figure \ref{Figure4}, the Mean-Variance Optimization (MVO) portfolio
emerges with the highest potential return accompanied by moderate risk.
Conversely, the Minimum Variance Portfolio (MVP) strategy exhibits the
lowest risk but correspondingly the lowest return. Both Equal Weight
(EW) and Risk Parity strategies offer a balanced combination of moderate
returns and risks, outperforming the S\&P 500 to some extent.

Figures \ref{Figure5} through \ref{Figure7} illustrate that the MVO
portfolio achieves the highest return but also carries the highest risk
relative to the S\&P 500 index. Meanwhile, the MVP portfolio features
the lowest risk but at the expense of the lowest return. The Risk Parity
portfolio showcases a moderate return with higher risk than the MVO
portfolio.

\begin{figure}[H]

{\centering \includegraphics{Finmetrics_Project_files/figure-latex/Figure5-1} 

}

\caption{Annualized Risk-return performance of portfolios relative to SP 500 since October 2008 \label{Figure5}}\label{fig:Figure5}
\end{figure}

\begin{figure}[H]

{\centering \includegraphics{Finmetrics_Project_files/figure-latex/Figure6-1} 

}

\caption{Annualized Risk-return performance of portfolios relative to SP 500 since December 2018 \label{Figure6}}\label{fig:Figure6}
\end{figure}

\begin{figure}[H]

{\centering \includegraphics{Finmetrics_Project_files/figure-latex/Figure7-1} 

}

\caption{Annualized Risk-return performance of portfolios relative to SP 500 since March 2020 \label{Figure7}}\label{fig:Figure7}
\end{figure}

Examining the portfolios' performance relative to the S\&P 500 during a
short-term volatile period in Figure \ref{Figure8}, suggests that the
portfolios outperform the S\&P index. Notably, the MVO portfolio
exhibits the highest annualized return with risk levels comparable to
those of the \(\frac{1}{N}\) and Risk Parity portfolios, albeit with the
latter two experiencing lower returns. In contrast, the MVP portfolio
performs less favorably among the portfolios.

During a short-term volatile period depicted in Figure \ref{Figure8},
all portfolios exhibit outperformance compared to the S\&P 500.
Particularly noteworthy is the MVO portfolio, which not only delivers
the highest annualized return but also maintains risk levels similar to
the \(\frac{1}{N}\) and Risk Parity portfolios, albeit with the latter
two experiencing lower returns. In contrast, the MVP portfolio lags
behind in performance among the portfolios considered.

\begin{figure}[H]

{\centering \includegraphics{Finmetrics_Project_files/figure-latex/Figure8-1} 

}

\caption{Annualized Risk-return performance of portfolios relative to SP 500 since January 2022 \label{Figure8}}\label{fig:Figure8}
\end{figure}

\hypertarget{index-statistics}{%
\subsection{Index Statistics}\label{index-statistics}}

This section comprehensively explores portfolio performance through an
in-depth analysis of various metrics, including returns, volatility,
expected shortfall, average drawdown, and performance during market
upswings and downswings. Notably, the assessment excludes the
examination of beta for the portfolios. The rationale for this omission
lies in the anticipated and confirmed beta values being less than one,
indicative of lower volatility compared to the benchmark due to better
diversification. Despite the exclusion of beta, other statistical
measures are deemed sufficient for providing a comprehensive overview of
the diverse portfolio performances.

The comparison in Table 1 reveals distinctive characteristics of each
strategy. The Minimum-Variance Portfolio (MVP) portfolio, while
exhibiting the lowest volatility and pain index, suggesting lower risk,
it also leads in terms of the adjusted Sharpe ratio compared to the
other portfolios, followed by the Mean-Variance (MVO) portfolio (seen in
Table 2), implying potential nefficiency in risk-adjusted returns,
aligning with conclusions by Chaves \emph{et al.}
(\protect\hyperlink{ref-chaves2011risk}{2011}).

The Risk Parity, along with the MVO, demonstrates superior annualized
returns, consistent with the findings of Maillard \emph{et al.}
(\protect\hyperlink{ref-maillard2010properties}{2010}). The Risk Parity,
MVO, and EW portfolios have a positive and greater than the benchmark
excess return statistic, whereas MVP has a negative excess return
statistic, it implies that RP, MVO, and EW have generated higher returns
than the S\&P 500 index over the same period, after adjusting for the
risk-free rate, whereas MVP has lagged in this regard

The S\&P 500, has the highest cumulative returns, but bears the highest
volatility and pain index. On the other hand, the Risk Parity portfolio,
with a higher adjusted Sharpe ratio than the S\&P 500, suggests
potential efficiency in risk-adjusted returns and also exhibits the
highest up-down ratio which indicates that the portfolio has a greater
ability to capture upside potential while minimizing downside risk. In
contrast the MVP has the lowest up-down ratio, followed by the equally
weighted portfolio which is unsurprising since these portfolios are
constructed to be more conservative. It suggests that these portfolios
have a greater focus on capital preservation and downside protection,
which also means it has a lower ability to capture upside potential.

The pain index, a measure of risk in losses, aligns with the drawdown
graph in Figure \ref{Figure3} and the average drawdown in Table 1. It
reaffirms that the S\&P 500 experiences the greatest drawdown, followed
by the Risk Parity portfolio, with MVP showing the lowest drawdown.

Examining tracking error and information ratio provides insights into
portfolio consistency and effectiveness. The MVO strategy stands out
with the lowest tracking error and the highest information ratio,
indicates that the portfolio has consistently generated excess returns
relative to the benchmark, taking into account the volatility of those
returns..MVP exhibits the highest tracking error. Correspondingly, the
information ratio suggests that MVP consistently underperforms the
benchmark on a risk-adjusted basis, aligning with its lower excess
returns and reaffirming the findings in Figure
\ref{Figure4}-\ref{Figure8}.

The modified Conditional Value at Risk (CVaR) assesses risk-adjusted
performance, representing the expected loss beyond the 95\% confidence
level. Therefore in the worst 5\% of the returns, your average loss of
the Risk Parity portfolio is lower than the S\&P 500 index. However,
again as expected, given the risk-averse nature of the portfolios, MVO
and MVP has a smaller expected average loss than the Risk Parity
portfolio, but the Equally weighted portfolio has a smaller average loss
than the MVO portfolio, and only marginally more than the MVP portfolio.

\setlength{\LTpost}{0mm}
\begin{longtable}{lrrrrr}
\caption*{
{\large Table 1: Long-Term Index Statistics: Relative to S\&P 500}
} \\ 
\toprule
Info\textsuperscript{\textit{1}} & SP 500 & Risk Parity & MVP & MVO & EW \\ 
\midrule\addlinespace[2.5pt]
\multicolumn{6}{l}{Historical} \\ 
\midrule\addlinespace[2.5pt]
Cum Returns & $520.6\%$ & $606.0\%$ & $486.1\%$ & $713.1\%$ & $526.5\%$ \\ 
Returns (Ann.) & $8.9\%$ & $9.5\%$ & $8.6\%$ & $10.2\%$ & $8.9\%$ \\ 
Returns Excess (Ann.) & $0.0\%$ & $0.6\%$ & $-0.3\%$ & $1.3\%$ & $0.0\%$ \\ 
SD (Ann.) & $15.1\%$ & $10.8\%$ & $7.9\%$ & $9.8\%$ & $8.7\%$ \\ 
Pain Index & 0.071 & 0.029 & 0.016 & 0.021 & 0.019 \\ 
Avg DD & $7.1\%$ & $4.3\%$ & $2.9\%$ & $3.6\%$ & $3.1\%$ \\ 
Tracking Error & $0.0\%$ & $11.0\%$ & $12.0\%$ & $10.8\%$ & $11.1\%$ \\ 
Information Ratio & - & 0.060 & -0.024 & 0.127 & 0.004 \\ 
Up-Down Ratio & 0.000 & 0.335 & 0.249 & 0.312 & 0.254 \\ 
Modified CVaR & -0.102 & -0.074 & -0.050 & -0.066 & -0.057 \\ 
\bottomrule
\end{longtable}
\begin{minipage}{\linewidth}
\textsuperscript{\textit{1}}Utilising the US 3-Month Libor Rate as a proxy for the risk-free rate\\
\end{minipage}

\setlength{\LTpost}{0mm}
\begin{longtable}{lrrrrr}
\caption*{
{\large Table 2: Adjusted Sharpe Ratio: Relative to S\&P 500}
} \\ 
\toprule
Info\textsuperscript{\textit{1}} & S\&P 500 & Risk Parity & MVP & MVO & EW \\ 
\midrule\addlinespace[2.5pt]
Adj. Sharpe Ratio & 0.669 & 0.478 & 0.635 & 0.58 & 0.568 \\ 
\bottomrule
\end{longtable}
\begin{minipage}{\linewidth}
\textsuperscript{\textit{1}}Utilising the US 3-Month Libor Rate as a proxy for the risk-free rate\\
\end{minipage}

\hypertarget{stratified-sample-periods}{%
\subsection{Stratified Sample Periods}\label{stratified-sample-periods}}

Tables 3 and 4 provide a more detailed exploration of the statistics
presented in Tables 1 and 2. Specifically, the analysis stratifies high
and low volatility periods concerning exchange rate movements (between
the South African Rand and the US Dollar), gold movements, and Brent
crude oil movements. The focus is on understanding how each portfolio
reacts relative to the S\&P 500 index during these different volatility
conditions.

The inclusion of exchange rate movement (US Dollar to Rand) is crucial
when constructing a portfolio with both local (South African) and global
(US) assets. Currency fluctuations can significantly impact portfolio
value, as changes in exchange rates affect the value of global assets
when converted back to the local currency. Foreign exchange exposure
involves the risk that fluctuations in exchange rates will influence the
value of a portfolio's assets, liabilities, or cash flows. For a
portfolio with both local and global assets, shifts in the exchange rate
between the Rand and the US Dollar can impact the value of US assets
when converted back to Rand.

In the context of gold and Brent crude oil literature, extensive
research demonstrates the spillover effects of the global crude oil and
gold markets on equity and bond markets\footnote{For interested readers,
  relevant literature includes Dai \& Kang
  (\protect\hyperlink{ref-dai2021bond}{2021}), Balcilar, Gupta, Wang \&
  Wohar (\protect\hyperlink{ref-balcilar2020oil}{2020}), Morrison
  (\protect\hyperlink{ref-morrison2019energy}{2019}), Kang, Ratti \&
  Yoon (\protect\hyperlink{ref-kang2014impact}{2014}), Akhtaruzzaman,
  Boubaker, Lucey \& Sensoy
  (\protect\hyperlink{ref-akhtaruzzaman2021gold}{2021}), Cohen \& Qadan
  (\protect\hyperlink{ref-cohen2010gold}{2010}), Lucey \& Tully
  (\protect\hyperlink{ref-lucey2003international}{2003}), Sadorsky
  (\protect\hyperlink{ref-sadorsky1999oil}{1999}), Chiang, Lin \& Huang
  (\protect\hyperlink{ref-chiang2013relationships}{2013}), Shahzad,
  Raza, Shahbaz \& Ali
  (\protect\hyperlink{ref-shahzad2017dependence}{2017}), Choudhry,
  Hassan \& Shabi
  (\protect\hyperlink{ref-choudhry2015relationship}{2015}), Morema \&
  Bonga-Bonga (\protect\hyperlink{ref-morema2020impact}{2020}), Gomes,
  Chaibi \& others (\protect\hyperlink{ref-gomes2014volatility}{2014}).
  However, these references won't be expanded upon in this analysis, as
  they fall outside the focus of this study.}.

Table 3 presents high volatility scenarios for the portfolios in
comparison to the S\&P 500 index.

In terms of risk-adjusted performance, the Risk Parity portfolio
consistently exhibits a lower adjusted Sharpe ratio compared to the
other portfolios. However, it still outperforms the S\&P 500 index.
Notably, the MVP portfolio stands out with the most promising
risk-adjusted returns among the portfolios.

It is interesting to observe that all the portfolios surpass the
benchmark index, aligning with the trends identified in Figures
\ref{Figure4} through \ref{Figure8}. This consistency in outperformance
suggests the effectiveness of the selected portfolio strategies across
various market conditions and reinforces their potential for delivering
superior risk-adjusted returns compared to the S\&P 500.

\setlength{\LTpost}{0mm}
\begin{longtable}{lrrrrr}
\caption*{
{\large Table 3: Fund Moments Comparison}
} \\ 
\toprule
Info\textsuperscript{\textit{1}} & SP 500 & Risk Parity & MVP & MVO & EW \\ 
\midrule\addlinespace[2.5pt]
\multicolumn{6}{l}{High Volatility Currency Movements} \\ 
\midrule\addlinespace[2.5pt]
Returns (Ann.) & $-1.3\%$ & $4.4\%$ & $7.3\%$ & $6.8\%$ & $6.4\%$ \\ 
Returns Excess (Ann.) & $1.2\%$ & $6.9\%$ & $9.9\%$ & $9.4\%$ & $9.1\%$ \\ 
Adj. Sharpe Ratio & -0.079 & 0.394 & 0.886 & 0.695 & 0.725 \\ 
Pain Index & 0.198 & 0.064 & 0.030 & 0.036 & 0.035 \\ 
Avg DD & $15.7\%$ & $11.3\%$ & $6.2\%$ & $6.7\%$ & $6.1\%$ \\ 
Tracking Error & $3.1\%$ & $13.8\%$ & $15.0\%$ & $13.7\%$ & $14.1\%$ \\ 
Information Ratio & Inf & 0.492 & 0.644 & 0.675 & 0.627 \\ 
Modified CVaR & -0.105 & -0.060 & -0.042 & -0.051 & -0.046 \\ 
\midrule\addlinespace[2.5pt]
\multicolumn{6}{l}{High Volatility Gold Movements} \\ 
\midrule\addlinespace[2.5pt]
Returns (Ann.) & $2.4\%$ & $10.3\%$ & $9.7\%$ & $10.8\%$ & $9.7\%$ \\ 
Returns Excess (Ann.) & $1.3\%$ & $9.1\%$ & $8.5\%$ & $9.6\%$ & $8.5\%$ \\ 
Adj. Sharpe Ratio & 0.145 & 0.856 & 1.115 & 1.023 & 1.018 \\ 
Pain Index & 0.119 & 0.036 & 0.018 & 0.025 & 0.022 \\ 
Avg DD & $23.8\%$ & $5.1\%$ & $3.6\%$ & $4.1\%$ & $3.7\%$ \\ 
Tracking Error & $2.7\%$ & $13.5\%$ & $14.4\%$ & $13.3\%$ & $13.5\%$ \\ 
Information Ratio & Inf & 0.686 & 0.599 & 0.731 & 0.638 \\ 
Modified CVaR & -0.103 & -0.061 & -0.042 & -0.052 & -0.047 \\ 
\midrule\addlinespace[2.5pt]
\multicolumn{6}{l}{High Volatility Brent Movements} \\ 
\midrule\addlinespace[2.5pt]
Returns (Ann.) & $-6.7\%$ & $4.0\%$ & $7.2\%$ & $5.7\%$ & $6.1\%$ \\ 
Returns Excess (Ann.) & $1.2\%$ & $12.8\%$ & $16.3\%$ & $14.7\%$ & $15.1\%$ \\ 
Adj. Sharpe Ratio & -0.369 & 0.341 & 0.866 & 0.546 & 0.662 \\ 
Pain Index & 0.324 & 0.077 & 0.026 & 0.044 & 0.035 \\ 
Avg DD & $51.6\%$ & $11.4\%$ & $6.7\%$ & $9.1\%$ & $6.3\%$ \\ 
Tracking Error & $3.1\%$ & $14.8\%$ & $16.5\%$ & $14.8\%$ & $15.4\%$ \\ 
Information Ratio & Inf & 0.800 & 0.906 & 0.912 & 0.899 \\ 
Modified CVaR & -0.111 & -0.065 & -0.042 & -0.054 & -0.047 \\ 
\bottomrule
\end{longtable}
\begin{minipage}{\linewidth}
\textsuperscript{\textit{1}}Utilising the US 3-Month Libor Rate as a proxy for the risk-free rate, except for Adjsuted Sharpe Ratio which uses a 0\% risk-free rate\\
\end{minipage}

Compared to the previous table, the performance of the portfolios (Table
4) are generally better under the low volatility scenarios than under
the high volatility scenarios. \setlength{\LTpost}{0mm}

\begin{longtable}{lrrrrr}
\caption*{
{\large Table 4: Fund Moments Comparison}
} \\ 
\toprule
Info\textsuperscript{\textit{1}} & SP 500 & Risk Parity & MVP & MVO & EW \\ 
\midrule\addlinespace[2.5pt]
\multicolumn{6}{l}{Low Volatility Currency Movements} \\ 
\midrule\addlinespace[2.5pt]
Returns (Ann.) & $12.5\%$ & $9.8\%$ & $7.6\%$ & $10.0\%$ & $8.3\%$ \\ 
Returns Excess (Ann.) & $0.0\%$ & $-2.4\%$ & $-4.3\%$ & $-2.2\%$ & $-3.7\%$ \\ 
Adj. Sharpe Ratio & 0.976 & 1.270 & 1.261 & 1.347 & 1.279 \\ 
Pain Index & 0.019 & 0.018 & 0.018 & 0.017 & 0.016 \\ 
Avg DD & $4.2\%$ & $3.0\%$ & $2.5\%$ & $2.8\%$ & $2.5\%$ \\ 
Tracking Error & $0.0\%$ & $9.3\%$ & $10.4\%$ & $9.4\%$ & $9.7\%$ \\ 
Information Ratio & - & -0.284 & -0.467 & -0.267 & -0.430 \\ 
Modified CVaR & -0.065 & -0.036 & -0.028 & -0.035 & -0.030 \\ 
\midrule\addlinespace[2.5pt]
\multicolumn{6}{l}{Low Volatility Gold Movements} \\ 
\midrule\addlinespace[2.5pt]
Returns (Ann.) & $7.6\%$ & $7.0\%$ & $6.5\%$ & $8.5\%$ & $6.8\%$ \\ 
Returns Excess (Ann.) & $0.0\%$ & $-0.5\%$ & $-1.0\%$ & $0.8\%$ & $-0.8\%$ \\ 
Adj. Sharpe Ratio & 0.450 & 0.775 & 0.905 & 1.009 & 0.861 \\ 
Pain Index & 0.046 & 0.031 & 0.018 & 0.021 & 0.021 \\ 
Avg DD & $9.7\%$ & $3.4\%$ & $2.4\%$ & $3.0\%$ & $2.6\%$ \\ 
Tracking Error & $0.0\%$ & $11.0\%$ & $12.4\%$ & $11.2\%$ & $11.5\%$ \\ 
Information Ratio & - & -0.052 & -0.086 & 0.078 & -0.072 \\ 
Modified CVaR & -0.093 & -0.047 & -0.039 & -0.043 & -0.042 \\ 
\midrule\addlinespace[2.5pt]
\multicolumn{6}{l}{Low Volatility Brent Movements} \\ 
\midrule\addlinespace[2.5pt]
Returns (Ann.) & $17.8\%$ & $11.0\%$ & $9.0\%$ & $10.9\%$ & $9.8\%$ \\ 
Returns Excess (Ann.) & $-0.1\%$ & $-5.9\%$ & $-7.6\%$ & $-6.0\%$ & $-6.9\%$ \\ 
Adj. Sharpe Ratio & 1.314 & 1.101 & 1.259 & 1.224 & 1.256 \\ 
Pain Index & 0.016 & 0.013 & 0.010 & 0.011 & 0.009 \\ 
Avg DD & $5.0\%$ & $2.3\%$ & $1.8\%$ & $2.6\%$ & $2.1\%$ \\ 
Tracking Error & $0.3\%$ & $9.4\%$ & $9.8\%$ & $9.1\%$ & $9.2\%$ \\ 
Information Ratio & -Inf & -0.736 & -0.918 & -0.777 & -0.887 \\ 
Modified CVaR & -0.061 & -0.051 & -0.035 & -0.044 & -0.038 \\ 
\bottomrule
\end{longtable}
\begin{minipage}{\linewidth}
\textsuperscript{\textit{1}}Utilising the US 3-Month Libor Rate as a proxy for the risk-free rate, except for Adjsuted Sharpe Ratio which uses a 0\% risk-free rate\\
\end{minipage}

\hypertarget{conclusion}{%
\section*{Conclusion}\label{conclusion}}
\addcontentsline{toc}{section}{Conclusion}

The analysis different portfolio startegies can be useful when investots
evaluate risk-adjusted performance that align with their risk tolerance.
For risk-averse investors prioritizing capital preservation, the MVP
strategy may be favored despite its negative excess return, given its
lower volatility and pain index. In contrast, risk-tolerant investors
seeking higher returns may lean towards MVO or EW, which exhibit higher
excess return statistics and upside potential.

\hfill

\newpage

\newpage

\hypertarget{references}{%
\section*{References}\label{references}}
\addcontentsline{toc}{section}{References}

\hypertarget{refs}{}
\begin{CSLReferences}{1}{0}
\leavevmode\vadjust pre{\hypertarget{ref-akhtaruzzaman2021gold}{}}%
Akhtaruzzaman, M., Boubaker, S., Lucey, B.M. \& Sensoy, A. 2021. Is gold
a hedge or a safe-haven asset in the COVID--19 crisis? \emph{Economic
Modelling}. 102:105588.

\leavevmode\vadjust pre{\hypertarget{ref-ardia2017impact}{}}%
Ardia, D., Bolliger, G., Boudt, K. \& Gagnon-Fleury, J.-P. 2017. The
impact of covariance misspecification in risk-based portfolios.
\emph{Annals of Operations Research}. 254:1--16.

\leavevmode\vadjust pre{\hypertarget{ref-balcilar2020oil}{}}%
Balcilar, M., Gupta, R., Wang, S. \& Wohar, M.E. 2020. Oil price
uncertainty and movements in the US government bond risk premia.
\emph{The North American Journal of Economics and Finance}. 52:101147.

\leavevmode\vadjust pre{\hypertarget{ref-bruder2012managing}{}}%
Bruder, B. \& Roncalli, T. 2012. Managing risk exposures using the risk
budgeting approach. \emph{Available at SSRN 2009778}.

\leavevmode\vadjust pre{\hypertarget{ref-chaves2011risk}{}}%
Chaves, D., Hsu, J., Li, F. \& Shakernia, O. 2011. Risk parity portfolio
vs. Other asset allocation heuristic portfolios. \emph{Journal of
Investing}. 20(1):108.

\leavevmode\vadjust pre{\hypertarget{ref-chiang2013relationships}{}}%
Chiang, S.-M., Lin, C.-T. \& Huang, C.-M. 2013. The relationships among
stocks, bonds and gold: Safe haven, hedge or neither. In
\emph{International conference on technology innovation and industrial
management}. 29--31.

\leavevmode\vadjust pre{\hypertarget{ref-choudhry2015relationship}{}}%
Choudhry, T., Hassan, S.S. \& Shabi, S. 2015. Relationship between gold
and stock markets during the global financial crisis: Evidence from
nonlinear causality tests. \emph{International Review of Financial
Analysis}. 41:247--256.

\leavevmode\vadjust pre{\hypertarget{ref-choueifaty2008toward}{}}%
Choueifaty, Y. \& Coignard, Y. 2008. Toward maximum diversification.
\emph{The Journal of Portfolio Management}. 35(1):40--51.

\leavevmode\vadjust pre{\hypertarget{ref-clarke2013risk}{}}%
Clarke, R., De Silva, H. \& Thorley, S. 2013. Risk parity, maximum
diversification, and minimum variance: An analytic perspective.
\emph{The Journal of Portfolio Management}. 39(3):39--53.

\leavevmode\vadjust pre{\hypertarget{ref-cohen2010gold}{}}%
Cohen, G. \& Qadan, M. 2010. Is gold still a shelter to fear.
\emph{American Journal of Social and Management Sciences}. 1(1):39--43.

\leavevmode\vadjust pre{\hypertarget{ref-costa2020robust}{}}%
Costa, G. \& Kwon, R. 2020. A robust framework for risk parity
portfolios. \emph{Journal of Asset Management}. 21(5):447--466.

\leavevmode\vadjust pre{\hypertarget{ref-dai2021bond}{}}%
Dai, Z. \& Kang, J. 2021. Bond yield and crude oil prices
predictability. \emph{Energy Economics}. 97:105205.

\leavevmode\vadjust pre{\hypertarget{ref-fisher2015risk}{}}%
Fisher, G.S., Maymin, P.Z. \& Maymin, Z.G. 2015. Risk parity optimality.
\emph{The Journal of Portfolio Management}. 41(2):42--56.

\leavevmode\vadjust pre{\hypertarget{ref-gomes2014volatility}{}}%
Gomes, M., Chaibi, A., et al. 2014. Volatility spillovers between oil
prices and stock returns: A focus on frontier markets. \emph{Journal of
Applied Business Research (JABR)}. 30(2):509--526.

\leavevmode\vadjust pre{\hypertarget{ref-kang2014impact}{}}%
Kang, W., Ratti, R.A. \& Yoon, K.H. 2014. The impact of oil price shocks
on US bond market returns. \emph{Energy Economics}. 44:248--258.

\leavevmode\vadjust pre{\hypertarget{ref-katzke2023portfoliocourse}{}}%
Katzke, N.F. 2023a.

\leavevmode\vadjust pre{\hypertarget{ref-katzke2023portfolio}{}}%
Katzke, N.F. 2023b. {[}Online{]}, Available:
\url{https://www.fmx.nfkatzke.com/posts/2020-08-07-practical-3/}.

\leavevmode\vadjust pre{\hypertarget{ref-ledoit2003improved}{}}%
Ledoit, O. \& Wolf, M. 2003. Improved estimation of the covariance
matrix of stock returns with an application to portfolio selection.
\emph{Journal of empirical finance}. 10(5):603--621.

\leavevmode\vadjust pre{\hypertarget{ref-lohre2012diversified}{}}%
Lohre, H., Neugebauer, U. \& Zimmer, C. 2012. Diversified risk parity
strategies for equity portfolio selection. \emph{The Journal of
Investing}. 21(3):111--128.

\leavevmode\vadjust pre{\hypertarget{ref-lohre2014diversifying}{}}%
Lohre, H., Opfer, H. \& Orszag, G. 2014. Diversifying risk parity.
\emph{Journal of Risk}. 16(5):53--79.

\leavevmode\vadjust pre{\hypertarget{ref-lucey2003international}{}}%
Lucey, B.M. \& Tully, E. 2003. International portfolio formation,
skewness and the role of gold. \emph{Skewness and the Role of Gold
(September 2003)}.

\leavevmode\vadjust pre{\hypertarget{ref-maillard2010properties}{}}%
Maillard, S., Roncalli, T. \& Teïletche, J. 2010. The properties of
equally weighted risk contribution portfolios. \emph{The Journal of
Portfolio Management}. 36(4):60--70.

\leavevmode\vadjust pre{\hypertarget{ref-markowits1952portfolio}{}}%
Markowits, H.M. 1952. Portfolio selection. \emph{Journal of finance}.
7(1):71--91.

\leavevmode\vadjust pre{\hypertarget{ref-merton1980estimating}{}}%
Merton, R.C. 1980. On estimating the expected return on the market: An
exploratory investigation. \emph{Journal of financial economics}.
8(4):323--361.

\leavevmode\vadjust pre{\hypertarget{ref-meucci2005risk}{}}%
Meucci, A. 2005. \emph{Risk and asset allocation}. Vol. 1. Springer.

\leavevmode\vadjust pre{\hypertarget{ref-meucci2009managing}{}}%
Meucci, A. 2009. Managing diversification. \emph{Risk}. 74--79.

\leavevmode\vadjust pre{\hypertarget{ref-morema2020impact}{}}%
Morema, K. \& Bonga-Bonga, L. 2020. The impact of oil and gold price
fluctuations on the south african equity market: Volatility spillovers
and financial policy implications. \emph{Resources Policy}. 68:101740.

\leavevmode\vadjust pre{\hypertarget{ref-morrison2019energy}{}}%
Morrison, E.J. 2019. Energy price implications for emerging market bond
returns. \emph{Research in International Business and Finance}.
50:398--415.

\leavevmode\vadjust pre{\hypertarget{ref-qian2005financial}{}}%
Qian, E.E. 2005. On the financial interpretation of risk contribution:
Risk budgets do add up. \emph{Available at SSRN 684221}.

\leavevmode\vadjust pre{\hypertarget{ref-sadorsky1999oil}{}}%
Sadorsky, P. 1999. Oil price shocks and stock market activity.
\emph{Energy economics}. 21(5):449--469.

\leavevmode\vadjust pre{\hypertarget{ref-shahzad2017dependence}{}}%
Shahzad, S.J.H., Raza, N., Shahbaz, M. \& Ali, A. 2017. Dependence of
stock markets with gold and bonds under bullish and bearish market
states. \emph{Resources Policy}. 52:308--319.

\leavevmode\vadjust pre{\hypertarget{ref-steinbach2001markowitz}{}}%
Steinbach, M.C. 2001. Markowitz revisited: Mean-variance models in
financial portfolio analysis. \emph{SIAM review}. 43(1):31--85.

\leavevmode\vadjust pre{\hypertarget{ref-RPP}{}}%
Vinícius, Z. \& Palomar, D.P. 2019. {[}Online{]}, Available:
\url{https://cran.r-project.org/web/packages/riskParityPortfolio/vignettes/RiskParityPortfolio.html\#risk-parity-portfolio}.

\end{CSLReferences}

\bibliography{Tex/ref}





\end{document}
